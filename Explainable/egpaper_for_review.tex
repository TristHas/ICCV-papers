\documentclass[10pt,twocolumn,letterpaper]{article}

\usepackage{iccv}
\usepackage{times}
\usepackage{epsfig}
\usepackage{graphicx}
\usepackage{amsmath}
\usepackage{amssymb}

% Include other packages here, before hyperref.

% If you comment hyperref and then uncomment it, you should delete
% egpaper.aux before re-running latex.  (Or just hit 'q' on the first latex
% run, let it finish, and you should be clear).
\usepackage[pagebackref=true,breaklinks=true,letterpaper=true,colorlinks,bookmarks=false]{hyperref}

\iccvfinalcopy % *** Uncomment this line for the final submission

\def\iccvPaperID{****} % *** Enter the ICCV Paper ID here
\def\httilde{\mbox{\tt\raisebox{-.5ex}{\symbol{126}}}}

% Pages are numbered in submission mode, and unnumbered in camera-ready
\ificcvfinal\pagestyle{empty}\fi
\begin{document}
	
	%%%%%%%%% TITLE
	\title{Assisting human experts in the interpretation of their visual process: A case study on assessing copper surface adhesive potency}%
	
	\author{
		Tristan Hascoet\\
		Kobe University\\
		{\tt\small tristan@people.kobe-u.ac.jp}
		\and
		Xuejiao Deng\\
		Kobe University\\
		{\tt\small dengxuejiao1005@yahoo.co.jp}
		\and
		Kiyoto Tai\\
		MEC Co., Ltd.\\
		{\tt\small tai295@mec-np.com}
		\and
		Yuji Adachi \\
		MEC Co., Ltd. \\
		\and
		Sachiko Nakamura\\
		MEC Co., Ltd.\\
		\and
		Tomoko Hayashi\\
		MEC Co., Ltd.\\
		\and
		Mari Sugiyama\\
		MEC Co., Ltd.\\
		\and
		Yasuo Ariki\\
		Kobe University\\
		\and
		Tetusya Takiguchi\\
		Kobe University\\
	}
	
	\maketitle
	%\thispagestyle{empty}
	
	
	%%%%%%%%% ABSTRACT
\begin{abstract}
Deep Neural Networks are often though to lack interpretability due to the distributed nature of their internal representations. In contrast, humans can generally justify, in natural language, for their answer to a visual question with simple common sense reasoning. However, human introspection abilities have their own limits as one often struggles to justify for the recognition process behind our lowest level feature recognition ability: for instance, it is difficult to precisely explain why a given texture seems more characteristic of the surface of a finger nail rather than plastic bottle.
In this paper, we showcase an application in which deep learning models can actually  help human experts justify for their own low-level visual recognition process: We study the problem of assessing the adhesive potency of copper sheets from microscopic pictures of their surface . Although highly trained material experts are able to qualitatively assess the surface adhesive potency, they are often unable to precisely justify for their decision process. We present a model that, under careful design considerations, is able to provide visual clues for human experts to understand and justify for their own recognition process. 
Not only can our model assist human experts in their interpretation of the surface characteristics, 
we show how this model can be used to test different hypothesis of the copper surface response to different manufacturing processes. 
\end{abstract}
	
	%%%%%%%%% BODY TEXT
\section{Introduction}

% Human explanation ability & deep learning flaws.
Humans are experts in communicating the reasoning process behind their answer to visual questions.
For instance, on typical Visual Question Answering (VQA) samples, 
human annotators are often able to convincingly justify, in natural language, the reason 
behind their answer to a certain visual question using simple common sense reasoning.
%Common sense includes spatial and causal relationships
In contrast, deep Learning models are often viewed as black box predictors lacking interpretability 
in the sense that existing tools often fail to explain the decision making process behind the model’s predictions.
For instance, a deep learning model trained end-to-end on a VQA dataset may be able to provide the same answer as its
human counterpart, but the process through which the model reaches this answer is entirely opaque.

% Limits of low-level introspection
While it is true that humans can justify for their answers on high level reasoning tasks, 
humans also often fail to explain the process behind their low-level feature recognition ability:
For example, precisely defining the nature of a specific texture 
(what are the defining features of a plastic or wooden surface?) 
or specific low-level part attributes exhibiting large intra-class variations 
(what is the defining features of a ``leg'' or a ``wing''?) is a very difficult task.
Humans constantly perform such low-level visual recognition tasks 
while being unable to precisely justify for their own recognition process.
	
% Current problem
In this paper, we present one very practical instance of such situation in the Printed Circuit Boards (PCB) industry, 
in which expert material scientists are tasked with assessing the adhesive potency of copper surfaces.
We propose a model that, under careful design considerations, is able to provide visual clues 
for human experts to understand and justify for their decision process.
	
% Ability of our proposed model.
The proposed model is designed so that a subset of its internal representations carry semantically meaningful 
information that can be visualized and easily interpreted by humans.
Providing these visual clues, however, comes with the cost of imposing additional constraints on the architecture,
which we found to degrade the model accuracy:
Indeed, we found that networks with unrestricted architectures, 
(which do not provide interpretable features)
perform better than network architectures restricted so as to provide 
semantically meaningful representations.
This is because, as we restrict the architecture of the model, 
we formulate an assumption on the impact of the manufacturing process 
on the surface statistics which may not hold in reality.
This result suggests an inherent trade-off between  
expressivity and the explainability in designing model architecture.
	
%the human experts, confirming their intuitions 
%and eventually shading light on their own decision processes and biases
% Test hypothesis using generalization performance as a metric of hypothesis validity
While the degradation of the model accuracy is problematic from a performance perspective,
it offers an interesting opportunity from an explainability perspective:
As the model accuracy degrades due to the inadequacy of the assumption 
made by the model architecture,
we can use the model accuracy as a proxy metric 
for the adequacy of different assumptions.
This allows us to quantitatively assess different assumptions regarding the impact 
of manufacturing processes on the copper surface. 
This may prove useful to quantify the impact of manufacturing process on cooper surface adhesive potency
and eventually help optimize the manufacturing process.
	
% Learn a hypothesis/theoretical model
%Going one step further we show how, given sufficient experimental data, the model
%can be augmented to automatically learn and formulate these hypothesis on itself.
% Conceptual contribution
In essence, the argument this paper is aiming for is as follows: 
although deep learning models lack the ``common sense reasoning'' abilities of humans,
and the powerful formalism of natural language to communicate and justify for their decision process, 
they can provide useful tools to visualize and explain low-level recognition processes.

% Practical contribution
In practice, the contribution of this paper is as follows:
\begin{enumerate}
\item  We formalize a segmentation procedure based on a probabilistic weak label segmentation framework.
\item  We introduce a formalism to show how the model accuracy can be used as a proxy metric to quantify the validity of different assumptions on the dataset.
\end{enumerate}
	
% Paper organization
The remainder of this paper is organized as follows:
In Section 2, we present some background information on the motivation for this project:
We start by discussing the importance of copper surface adhesive potency,
and detail the dataset used in our experiments. 
Section 3 details our contribution.
Section 4 briefly relates our work to different research topics 
and Section 5 presents the results of our experiments.
Finally, Section 6 further discusses the relevance of our results, insisting on the limitations of our assumptions to conclude this paper.
	
%------------------------------------------------------------------------
\section{Background}

Printed circuit boards (PCBs) are an integral part of a wide variety of electronic devices, 
including industrial and household appliances (e.g. TV and PC), 
mobile communication devices and automobiles. 
PCBs play an important role in electrical connection between electronic components. 
Copper has been used in the PCBs industry as the conducting material, 
and the electric copper circuits are isolated from each other by insulators (solder resist, prepreg etc). 
A multilayered PCBs have a laminate having a plurality of electroconductive 
layers with insulating layers interposed therebetween. 
So there are many interfaces related to the copper and resins in PCBs. 
Figure 1 illustrates the organization of such an electric circuit.


\begin{figure}[h]
\centering
\includegraphics[width=0.9\linewidth]{"./figures/Figure1"}
\caption{
Illustration of a Printed Circuit Board. 
Copper circuits are made with various insulators for several purposes.  
}
\end{figure}


Since PCBs have been required to have higher heat-resistant properties in recent years, 
copper surface treatment technologies have performed a more and more important role in the manufacturing process. 
This is because they can enable PCBs to maintain high copper to resin adhesion even under harsh conditions. Copper surface treatment (copper surface roughening) offers one of the most effective ways to increase adhesion of the interfaces. Copper surface roughening has been widely used for the purpose of increasing adhesion of copper to resins. It produces a unique surface topography which enhances the mechanical bonding of copper to resins, as illustrated in Figure 2.

\begin{figure}[h]
\centering
\includegraphics[width=0.9\linewidth]{"./figures/Figure2"}
\caption{
Illustration of a copper roughness surface. 
(Left) a perfectly smooth surface provides small adhesive surface 
as the interface between copper, in brown, and the resin, in green is minimal. 
(Right) a rough surface provides a larger surface at the interface of the resin. 
Larger contact surface areas provide higher adhesive potency.
}
\end{figure}

In very broad terms, rough surface allow for stronger bonds 
as the asperities of the surface provide a wide range of adhesive surface area and an anchoring effect. 
In contrast, smooth surfaces don’t provide such effects so that they have lower adhesive potential. 
In the remaining of this paper, we will refer to the potential bonding strength of a copper surface as its ``adhesive potency''.
It is also important to note that the adhesive potency of a copper surface is related to its ``roughness'', 
which is observable at the microscopic scale.

Electronic substrate manufacturers have developed advanced manufacturing processes to shape the surface of copper sheets in order to increase their adhesive potency. 
This is typically achieved by applying an etching solution on the copper surface.
Being able to accurately assess the adhesive potency of a copper surface would 
allow to further optimize manufacturing processes to increase the reliability of electronic devices.
However, assessing the adhesive potency of a copper surface is a complex task, even for the most expert practitioners. 
Hence the motivations of this study is two fold: 
First we aim to automate the evaluation of a copper sheet adhesive potency from microscopic imaging of its surface. 
Second, we aim to better understand the process through which  
Towards this goal, we built a dataset of microscopic images of copper surfaces, which we detail below.

We imaged copper surfaces using Scanning Electron Microscopy (SEM) at a resolution of $100$ nm.
To investigate the impact of different manufacturing processes on the copper surface, 
we applied 16 different etching solutions, with decreasing etching power, to the copper surface.
For each of these solutions, we captured 50 SEM images of $960 \times 1280$ pixels so that the full dataset
consists of $800$ ($16 \times 50$) images.
%The etching power of each solution was degraded by a constant factor $\delta$ between each 
Each image is annotated with a label $t$ corresponding to the etching solution used to shape the copper surface.
Each solution was obtained by submitting the original solution $t=0$ to an extreme stress test for a period of time $t$.
Hence, we know that for all images of copper surfaces with label $t$ show higher adhesive potency than the images labelled with $t' > t$. 
However, we do not know the \textit{exact} impact of the stress test on the surface adhesive potency.

\begin{figure}[h]
\centering
\includegraphics[width=0.9\linewidth]{"./figures/Figure3"}
\caption{
%Need a cleaner picture: 
%todo change font
%todo change label name to t
%todo Adjust picture size to be similar
Illustration of a few images from the dataset. 
(Top) Full images. 
(Down) Zoomed-in areas of $200 \times 200$ pixels. 
(Right) Sample image of label $y=1$. 
(Left) Sample image of label $y=10$. 
The difference between both images are minimal to an untrained eye.
Precisely defining the visible difference with words is a difficult task.
}
\end{figure}

\section{Method}

\subsection{Dataset and Notations}
%todo use the correct number of weeks
We denote the dataset described above as $\mathcal{D}=\{(x_i,t_i) | i \in [1,800] \}$ where $x$ denote images $x \in \mathbb{R}^{H \times W}$ and labels $y \in [0,16]$ corresspond to the time of stress test applied.
We split the dataset $\mathcal{D}$ into a training $\mathcal{D}_{tr}$, validation $\mathcal{D}_{val}$ and test $\mathcal{D}_{te}$ set so that the number of images $x$ per label $y$ in each set are 40 for training set, and 5 for the validation and test sets.

\subsection{Baseline}

We start by establishing a strong baseline for our study.
The baseline architecture follows standard convolutional network designs for image classification.
This architecture, illustrated in Figure 4, is made of several residual blocks 
sequentially interleaved with max pooling operations.
Each residual block consists of $n$ repetitions of a sequence of 
$3 \times 3$ Convolution, Batch Normalization and ReLU layers,
followed by a residual skip connection.
We set $N$ residual blocks between every max pooling layer
and we denote by $d$ the number of pooling layers.
Hence, the full depth $D$ of the network (in number of convolution layers) 
is given by $D=d \times n \times N +1$, where the term $1$ corresponds to the
initial $3 \times 3$ Convolution layer happening before the first pooling operation.
Finally, the top layer of the network is made of a global average pooling layer 
followed by a linear softmax layer with output dimension 15 corresponding to our number of classes.
For simplicity and contrary to standard practices, we keep the number of channels $c$ constant 
in all layers of the network.

\begin{figure}[h]
\centering
\includegraphics[width=0.9\linewidth]{"./figures/Figure4"}
\caption{
Illustration of our baseline architecture.
With our modular architecture definition, 
the architecture is fully defined by parameters $n$,$N$,$c$ and $d$.
}
\end{figure}

With this parameterization, our network is fully specified by the four hyper parameters $c$, $d$, $n$ and $N$.
We performed a grid search over these hyper parameters to select the best performing architecture.
The details of this architecture search are given in the experiment section,
and, as we shall see then, the best performing architecture performs significantly better than human experts.
However, the decision process through which this model reaches such a high accuracy is entirely opaque
as the distributed nature of the model's internal representations provides little interpretability.
The remainder of this section details our attempt to design an architecture that can provide 
useful explanations of the process through which high accuracy recognition can be performed.

\subsection{Assumptions}
% Basic idea
The basic idea behind our method is similar to the visual insights provided by the
visualization method of \cite{} and the attention mechanisms in visual models 
for image captioning and question answering tasks:
We would like to evaluate the contribution of each input pixel to the final classification decision. 
%Having such a visual explanation could help us explain the model's decision process 

To do so, we modify our initial problem formulation into a segmentation task:
Given an input $x \in \mathbb{R}^{H \times W}$, we want to design a model that outputs 
a segmentation mask $s \in \mathbb{R}^{H \times W}$ assessing the contribution of each inidividual pixel to the output adhesive potency score (i.e., the output class $y$).
Training a typical segmentation model for this task would require ground truth 
segmentation masks $s$ for each image $x$ of the training set.
However, manually annotating ground truth segmentation masks for this task is not feasible, 
as human experts are not able to provide such fine-grained annotations.
Instead, we have to train the segmentation given a single ground truth label $t$ per image.
In order to do so, we make the following assumption:

\textbf{Assumption}: 
For a given image $x$ with label $t$, 
the \textit{pixel-wise} values of the true (unknown) segmentation mask $s$
follow a spatially stationary binomial distribution whose expected value, 
averaged over the spatial dimensions of $x$, is given by $t$ following:

\begin{equation}
  s_{hw} \sim \mathbb{B}(f(t)), \forall h,w,t \in H \times W \times T
\end{equation}

in which we introduced a target function $f$, which we shall discuss in Section 3.6.

%For a given image, we suppose the existence of an unknown 
%ground truth binary that assigns a value of 1 to regions of
In other words, we suppose that there exists a true binary segmentation map
$s$ assigning to each individual pixel of $x$ a binary adhesive potency score:
$s$ takes 0 values in regions of the surface providing low adhesive potency (i.e. smooth copper surface areas)
and 1 values in regions of the surface providing high adhesive potency (i.e. rough copper surface areas).
The adhesive potency score of an entire image $x$ is thus given by the average of the pixel-wise values,
and this average value is uniquely defined by the image label $t$.

\begin{figure}[h]
\centering
\includegraphics[width=0.9\linewidth]{"./figures/Figure5"}
\caption{
%todo change y with t
Illustration of binary segmentation masks. 
White pixels represent areas of high adhesive potency
and black surface represent areas of low adhesive potency.
We make the assumption that the ratio of white surface is 
constant for different samples (top and bottom) with equal label $t$.
The exact ratio is defined by the target function $f$.
}
\end{figure}

Figure 5 provides a visual illustration of this idea.
For each image, we suppose the existence of such binary mask $s$
quantifying the adhesive potency of local areas of the surface.
The average value of the binary mask $s$ amounts to the ratio of the copper surface covered in white 
(i.e. with value 1, corresponding to high adhesive potency).
Our assumption means that this ratio stays constant for different 
images $x$ sharing a similar label $t$,
and takes a value given by the target function $f$.

\subsection{Architecture}

% 
%We modify the baseline architecture presented in Section  to output binary segmentation maps.
In this section, we present the architecture used 
to compute binary segmentation masks from input images.
Our architecture extends the baseline architecture presented in Section 3.1,
with an ascending path that progressively upsamples the output
of the descending path, similar to the UNet architecture \cite{ronneberger2015u} and illustrated in Figure 6.

Residual modules of the axcending path are the exact symmetric of their anolog in descending path.
The ascending path uses bilinear upsampling layers 
instead of the max pooling layers of the descending path.
Different from the UNet architecture, and following previous works \cite{xxx},
we merge the outputs of the descending path modules with the inputs of the ascending path 
by summation, instead of concatenation. 
We alos use valid convolutions to preserve the spatial resolution of the output.
Finally, we add a sigmoid layer at the top of the network in order to bound the output values between 0 and 1.

\begin{figure}[h]
\centering
\includegraphics[width=0.9\linewidth]{"./figures/Figure6"}
\caption{
Illustration of the segmentation model architecture.
This architecture follows standard practice in 
UNet-like segmentation architectures.
}
\end{figure}

\subsection{Loss Function}

Given a segmentation model $M_{\theta}$, 
with weight parameters $\theta$, 
and an input image $x$,
we denote the average of the model output by $m_{\theta}(x)$:

\begin{subequations}
\begin{align}
M_{\theta}&: \mathbb{R}^{H \times W} \rightarrow [0,1]^{H \times W} \\
\tilde{s}&= M_{\theta}(x) \\
m_{\theta}&:  \mathbb{R}^{H \times W} \rightarrow [0,1]\\
m_{\theta}(x)&=  \frac{1}{HW} \sum_{h,w} \tilde{s}_{hw}
\end{align}
\end{subequations}

We can then train the model by regressing the average value of the segmentation mask to the
target label given by $f(t)$. 
Given a training dataset of labeled samples $\mathcal{D_{tr}}={(x_i, t_i)}$,
learning is thus done by minimizing the following loss function over the model's parameters $\theta$:

\begin{equation}
\theta^* = argmin_{\theta} \mathbb{E}_{x,t \in \mathcal{D}_{tr}} ||m_{\theta}(x) - f(t) ||^2 
\end{equation}

However, as we shall see in the next section, 
the target function $f$ represents an unknown ideal function,
so we do not have access to the actual values of $f(t)$.
Instead, we will approximate $f$ with a known hypothesis function $g \approx f$,
so that the actual training loss used in our experiments is:

\begin{equation}
\theta^* = argmin_{\theta} \mathbb{E}_{x,t \in \mathcal{D}_{tr}} ||m_{\theta}(x) - g(t) ||^2 
\end{equation}

\subsection{Target Function}

% Intro
In section 3.3, we have made the assumption 
that labels $t$ uniquely define the expected value 
of the ground-truth binary segmentation masks $s$ through a target function $f$.
In this section, we discuss the role of this target function.

% What does this target mean?
The target function $f(t)$ describes the evolution of the copper surface adhesive potency with time $t$.
More precisely $f$ defines the evolution of the \textit{ratio of adhesive surface area} with time (see Figure 5).

% f is an ideal function that we estimate
However, $f$ is an ideal, unknown function, of which we have only supposed the existence.
We do not know the value taken by $f(t)$ for a given $t$ because we 
do not know the exact impact of the manufacturing process on the copper's surface characteristics.
We thus introduce a known hypothesis function $g$ to approximate the ideal target function $f$.
$g$ expresses our belief of what the values taken by the true function $f(t)$ are.
Although we do not know the exact values taken by $f$, we know several of its characteristics,
which we can use to reduce the search space of hypothesis functions $g$:

% f is monotonically decreasing
In Section 2, we have established that, for all $t$, 
copper surfaces with label $t' > t$ should have lower adhesive potency than copper surfaces of label $t$.
Hence, $f$ should be a monotonically decreasing function of time,
and we can thus restrict our search of hypothesis function $g$ to monotonically decreasing function.
For example, the simplest of such function would be the linear function, 
taking linearly decreasing values from 1 to 0,
which we start by analyzing in the experiment section.

% The generalization error reflects the accuracy of g's apporximation
%todo Sharper explanation
Second, as $f$ is assumed to describe the true evolution of the copper surface statistics with $t$,
a universal function approximator should be able to perfectly learn its values, given enough training data.
Hence training an ideal 

training an ideal model with the supervision signal provided by $f$ should yield low test errors.
Hence training a segmentation model with a hypothesis $g \approx f$ should lead to low test errors.
This means that we can use the error of the model on the held out validation dataset as a proxy measure on
the validity of the hypothesis function $g$.

% We can learn f
In the experiment section, we evaluate the model errors with different hand-crafter hypothesis functions $g$,
and show that the model yields large errors when trained on a poorly choses hypothesis function $g$.
This motivates our idea that the model test error may be used as a proxy metric of the quality of the hypothesis function $g$.
In future work, we aim to jointly learn the hypothesis function $g$ with the model parameters $\theta$ to automatically
formulate hypothesis


\section{Related Work}

% Intro
We identify three different lines of research that share similarities with our study:
Explainability of learned visual representations, 
weak supervision of segmentation models and 
machine learning applications for material science.
Our can be seen as a weakly supervised approach to provide material 
experts with strong supports for interpretable decision, 
which falls at the intersection of these three research lines.
We briefly present some of these works in the following subsections

\subsection{CNN Interpretability}
% intro
The actual processing performed by deep models is hard to interpret:
because of the distributed nature of the model's internal representation,
it is difficult to assign a useful meaning to each individual unit meaning.
This is problematic as model failure cases are very hard to investigate and justify.
Hence, an interesting line of work is researching for tools to interpret neural network processes.
We can see two approaches in this line of work:
one is focused on visualization and the ther one focused on generating natural language explanations.
 
% NLP
%todo
On the side of natural language explainability, Park \textit{et al.} \cite{park2016attentive} jointly trained to answer and justify for their
answers on a visual question answering task. 
They combine visual attention maps and natural language generation to bring interpretability to the model's output.
 
% Visualization
More related to our work is the line of work focused on visualizing neural network hidden activations
% Jason yo
% Jason Yosinski et al.,2015,understanding neural network through deep visualization
Yosinski \textit{et al.} \cite{yosinski2015understanding} provide two useful tools for visualizing and interpreting neural nets. 
One is to visualize  the activations produced on each layer of a trained convnet as it 
processes an image or video, and the other enables visualizing 
features at each layer of a DNN via regularized optimization in image space.
% activation atlas https://distill.pub/2019/activation-atlas/

More recently, Carter \textit{et al.} \cite{xxx} propose an explorable activation atlas of vision model's learned feature by using feature inversion to visualize millions of activations from an image classification network
Their technique provides insights regarding the network's conceptual representations.
% attention models for image captioning and visual question answering
%Yashi Goyal et al., CVPR 2017, Marking the V in VQA Matter:Elevating the Role of Image Understanding in Visual Question Answering

% Original work on visual attention
Visual attention \cite{xxx} also helps in explaining the process through which model's 
outputs are computed by providing visual clues as to what regions of the input space 
contribute the most to the final decision. 
% show attend and tell
% other visual models in question answering and image captioning
These have been used to investigate model's operations on image captioning \cite{xxx} and VQA \cite{xxx} tasks,
and is similar to the idea behind our segmentation model.

\subsection{Weakly Supervised Segmentation}
%todo
%% Segmentation from lazy labels.
An other point we should pay attention to is different labeling strategies in conventional supervised learning.
Typically speaking, it's often assumed that each instance is associated with one single label. However, there should usually be more than one labels for one instance in real-world tasks, if it is multi-label learning.

% Detection with weak labels
%%Matthew B.Blaschko et al., 2010,Simultaneous Object Detection and Ranking with Weak Supervision

%%Ross B. Girshick et al., 2011,Object Detection with Grammar Models
%%Bharath Hariharan et al.,2014,Simultaneous Detection and Segmentation
%%Hakan Bilen et al.,2016,Weakly Supervised Deep Detection Networks

% segmentation weak label https://arxiv.org/pdf/1904.01636.pdf
%%Eugene Vorontsov et al., 2019, Boosting segmentation with weak supervision from image-to-image translation
Eugene et al. propose a semi-supervised framework that employs image-to-image translation between weak labels.
% Semi-supervised learning for object detection or semantic segmentation
%%Yuxing Tang et al.,2016, Large Scale Semi-supervised Object Detection using Visual and Semantic Knowledge Transfer
Yuxing Tang et al. build a similarity-based knowledge transfer mode trying to investigate whether knowledge about visual and semantic similarities of object categories can help improve the peformance of detectors trained in a weakly supervised setting.

% Our work = Merge segmentation from lazy labels and scientific discovery
%% Rihuan Ke et al., 2019, A multi-task U-net for segmentation with lazy labels
The most related work with ours is Rihuan Ke's\cite{ke2019multi} in which they present a semi-supervised learning strategy for segmentation with lazy labels and develop a multi-task learning framework to integrate the instance detection, separation and segmentation within a deep neural network.

\subsection{Machine Learning for Material Science}
%todo
Deep learning has showed remarkable success on many important 
learning problems in chemistry, drug discovery, biology and materials science.

In the field of materials science, deep neural networks have also been receiving inreasing attention and have achieved great improvements, for example, in material property prediction and new materials discovery for batteries.

CNNs have also been used for defect detection on microscopic images of various material surfafces \cite{}.
% Material sciences
% Discover new materials or properties (messaging passing deep networks by gilmer)
%%Justin Gilmer,2017, Neural Message Passing for Quantum Chemistry
%A general framework for supervised learning on graphs called Message Passing Neural Networks (MPNNs) is widely used on neural computation for predicting quantum states of molecules.
% Drug discovery: Find molecules useful by predicting their property
%%Dibyendu Dana et al.,2018, Deep Learning in Drug Discovery and Medicine;Scratching the Surface
%%Jessica Vamathevan et al.,2019, Applications of machine learning in drug discovery and development
%Drug discovery is also benifiting from the development of artificial intelligence technology, which is automating the invention of new chemical entities and the mining of large databases, in drug design and molecular medicine field.  
% Protein folding
%Deep learning approaches such as DeepMind's AlphaFold are helpful for scientists to deal with the "protein folding problem", not only predicting the intricate 3D structure of a protein but also predicting the physical properties of a protein structure, and are making significant progress on one of the core challenges in biology.
% THIS ONE New Materials for batteries
%% 

Beyond the material sciences, we note a growing interest in applying deep learning techniques
for scientific discovery. 
This perhaps best examplified by the impressive successes of AlphaFold \cite{} in protein folding estimation,
or the Celeste \cite{} project which catalogued celestial objects of visible universe.

Our work, while much more modest in its scale, 
shares the characteristic of using vision models to unravel
the unerdlying principle of material reactions to chemical treatments.

\section{Experiments}

\subsection{Classification results}
% 
We start by evaluating the baseline classification model described in Section 3.3 and illustrated in Figure 4.
The hyper parameters $n$, $N$, $d$, and $c$ of the classifier architecture 
were obtained by a grid search within the limits of a 12GB Nvidia GPU memory size .
The model was trained on the training dataset for 200 epochs using the Adam optimizer with default parameterization.

We compare the classification outputs of the baseline model to those 
of an expert material scientist on a blind test.
The human expert was given the sample images of the training set to practice,
and we evaluated the expert's answers on the samples of the test set.

\begin{figure}[h]
	\centering
	\includegraphics[width=0.9\linewidth]{"./figures/Figure7"}
	\caption{
		Results of the human evaluation.
	}
\end{figure}

Figure 7 shows the results achieved by the model and Figure 8 shows the results achieved by our baseline model.
As can be seen in these figures, the model tends predict the manufacturing 
process more accurately than the human evaluator.
However, these results should be taken with a grain of caution 
as the human expert was given little to practice on this specific dataset,
while the model was selected as the best performing baseline from an extensive parameter search.
We plan on re-conducting the human evaluation in the updated version of this paper.

\begin{figure}[h]
	\centering
	\includegraphics[width=0.9\linewidth]{"./figures/Figure8"}
	\caption{
		Results of the model evaluation.
	}
\end{figure}

Interestingly, however, the human expert seems to accurately recognize 
the surfaces of highest adhesive potency as seen in the upward trends of his results 
for low $t$ values. For example, he easily identified the surface for $t=0$.
On the other hand, his guesses for low adhesive potency are much more random.

This is in stark contrast with the classifier accuracy, which perfectly identifies
the low adhesive potency surfaces (high $t$ values), but consistently misclassifies 
surfaces of high adhesive potency (i.e.; for $t=1$ and $t=2$).

\subsection{Visualization of Segmentation Results}

% Visualize
To motivate our results, Figure 9 illustrates an output of the segmentation model on a small patch of an input image $x$.
To a novice observer, the model seems to assign lower adhesive potential to smoother regions of the input.
In future work, we plan on further exploring these visualizations with human experts to better understand the 
patterns characterizing the surface adhesive potency.

\begin{figure}[h]
	\centering
	\includegraphics[width=0.9\linewidth]{"./figures/Figure9"}
	\caption{
		Visualization of the segmentation model output.
	}
\end{figure}

\subsection{Investigation of the hypothesis function}
%todo
In this section, we question the role of the hypothesis function on the accuracy of the model.
As illustrated in Figure 10, we investigate the simplest possible hypothesis, $f(y)=y$.
This hypothesis function formulates the idea that for $t=0$, 
all the  segmentation mask values should be 0, 
so that the whole surface area should have strong adhesive potency, 
while for $t=14$, all the segmentation mask values should be $1$, so that 
the whole surface area should have low adhesive potency.
For each $t$ between 1 and 14, this hypothesis function defines a linear increase
of the low adhesive potency surface ratio.

\begin{figure}[h]
	\centering
	\includegraphics[width=0.9\linewidth]{"./figures/Figure10"}
	\caption{
		Results of the segmentation model evaluation.
	}
\end{figure}

Figure 10 compares the hypothesis function to the actual model output on the training, 
validation and test set.
First, we observe no overfitting as all datasets yield similar average results.
Second, we observe that instead of following a linear trend from 0 to 1, 
the surface ratio seems to evolve more similarly to a squared root or logarithmic function
with a sharp increase in value for low $y$ and a slower increase in higher values of $y$.
It is interesting to note that the same behavior is also observed on samples of the training set, 
for which the model was explicitly trained to reproduce the linear hypothesis function.

This result suggests that the surface ratio (i.e. the smoothing effect of the stress test on the copper surface) may increase sub-linearly with time, either as a square root or logarithmic trend.
Investigating the impact of different hypothesis function is an 
interesting question we will consider in future work.

\section{Discussion \& Conclusion}
%todo
% Summary of the paper

In this paper, we have argued that deep neural networks can be used to 
help explain the process behind low-level recognition tasks.
We have focused on the task of assessing the adhesive potency 
of copper surfaces in the context of PCB manufacturing.

This is an interesting task to showcase the limitations of human's ability to explain their visual process
as it is a very low level recognition task for which trained experts can provide qualitative guesses while
not being able to fully justify for their guess.

On this task, we have first shown that an unrestricted classifier 
architecture can outperform a human expert in accuracy.
However, this architecture does not bring us any insight 
into \textit{why} a given copper surface should be classified 
as a high or low adhesive potency.

To shed some light into the decision process of the model,
we proposed to cast this problem as a a segmentation problem, 
in the model assigns a binary score to each pixel indicating wehter this
pixel belongs to a high or low adhesive power local area of the surface.
Visualizing the segmentation output of the model may prove useful for 
human experts to better understand the characteristics of high and low 
adhesive potency surfaces.

Finally, we developed a weak label training procedure to train this segmentation model.
Our procedure relies on a hypothetical relationship between the manufacturing process and the
copper surface roughness.
By training the network on the hypothesis that the surface ratio of low adhesive potency increases
linearly with the stress test time, we observed that the model output follows a log-like evolution.

% Limits of the assumption
However, the result of this last experiment should be taken with caution as this study is still in a very preliminary stage.
In particular, our training procedure relies several assumptions that most likely do not hold in reality: 
In particular, we assumed that each individual area could be represented by a single binary value defining its adhesive potency
while, in reality, the adhesive potency of local areas are most likely not binary in nature.

Nevertheless, these preliminary results are encouraging, and we will continue our analysis in future work.

{\small
	\bibliographystyle{ieee}
	\bibliography{egbib}
}

\end{document}
